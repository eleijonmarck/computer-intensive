\subsection*{d)}
In this exercise, we are supposed to investigate the sensitivity of the posteriors when we vary the hyperparameter $\beta$. Since we do not have a "correct" answer for all of the parameters, we will investigate the sensitivity by calculating the variance of the parameters when drawing from the posteriors several times. We will investigate what happens for $\beta = 0, 1, 5, 20$ with a constant $\rho = 0.14$. In order to investigate the sensitivity we chose simulate the chain 100 times for each value of $\beta$ with a burn--in of 10,000 samples, 25,000 samples and two breakpoints. The variance for each of the parameters was then calculated and are found in the table below. Each of the columns represent the variance of that parameter.

\begin{table}[H]
\centering
\begin{tabular}{|c|c|c|c|c|c|c|}
\hline
Parameters & $\theta$ & $t_2$ & $t_3$ & $\lambda_1$ & $\lambda_2$ & $\lambda_3$  \\ \hline
$\beta = 0$ & $1.7\cdot 10^{-4}$ & 0.14 & 4.21 & $1.5 \cdot 10^{-5}$ & $8.3 \cdot 10^{-4}$ & $3.6\cdot 10^{-4}$ \\ \hline
$\beta = 1$ & $1.2\cdot 10^{-4}$ & 0.16 & 5.23 & $1.3 \cdot 10^{-5}$ & 0.0012 & $4.2 \cdot 10^{-4}$ \\ \hline
$\beta = 5$ & $3.3 \cdot 10^{-5}$ & 0.43 & 7.44 & $4.58 \cdot 10^{-5}$ & 0.0031 & $4.6 \cdot 10^{-4}$ \\ \hline
$\beta = 20$ & $2.23 \cdot 10^{-6}$ & 0.72 & 7.55 & $5.9 \cdot 10^{-4}$ & 0.0068 & $9.8 \cdot 10^{-4}$ \\ \hline
\end{tabular}
\end{table}

As we can see in the above table we have the lowest overall variance of the parameters for $\beta = 0$ followed up by $\beta = 1$ etc. This implies that the estimates will be less prone to vary the lower the value of $\beta$ is. One thing to note is that the variance of $\theta$ is improved the greater the value of $\beta$ where as the other parameters' variances are increased. One must thus use a sufficienlty large enough $\beta$ in order to get "accurate" values of $\theta$ where as you need a small enough value to get accurate values of the parameters. \\ \\ It should be noted that we only simulated the chain 100 times, which might not be enough in order to get stable results. But since we used 25,000 samples for each simulation we consider these results to be somewhat accurate. We were unfortuneately not able to simulate more since the simulations are rather slow, but we simulated the chain a couple of times for lower amount of times and got almost the same results.

%In this exercise we are supposed to investigate the behaviour of the chain when we vary the hyperparameter $\beta$. We investigate what happens for $\beta = 20$ and $\beta = 0$. When we use $\beta = 20$ we get very varying estimations of the breakpoints from one simulation to another, whereas the intensities and $\theta$ are somewhat consequent. The estimations of $\theta$ and the intensities are relatively constant. Thus for large values of $\beta$ the breakpoints are effected. \\
%\\
%However, if we set $\beta = 0$ the estimations are consistent in between simulations, meaning that we do not get results that vary notably between simulations. The results are basically the same as in \textbf{c)}, but the estimation of the first breakpoint when using three breakpoints is more "exact" in the sense that it sets the first breakpoint to $\approx 1890$. \\ The conclusion is that the posteriors are sensitive if one sets the hyperparameter to be too big. Whereas the posteriors become less prone to instability for small values of $\beta$, but not necessarily more correct. It is thus important not to set $\beta$ to be too big.

